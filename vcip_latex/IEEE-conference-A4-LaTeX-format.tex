% IEEE Paper Template for A4 Page Size
% Sample Conference Paper using IEEE LaTeX style file for A4 page size.
% Copyright (C) 2006-2008 Causal Productions Pty Ltd.
% Permission is granted to distribute and revise this file provided that
% this header remains intact.
% 
% REVISION HISTORY
% 20080211 changed some space characters in the title-author block
%
\documentclass[10pt,conference,a4paper]{IEEEtran}
\usepackage{times,amsmath,epsfig}
\usepackage{epstopdf}
\usepackage{amsfonts}
%
\title{Evenness-controllable point cloud simplification via graph filter}
%
\author{%
% author names are typeset in 11pt, which is the default size in the author block
%{First Author{\small $~^{\#1}$}, Second Author{\small $~^{*2}$}, Third Author{\small $~^{\#3}$} } %  Removed for anonymous submission
{}
% add some space between author names and affils
\vspace{1.6mm}\\
\fontsize{10}{10}\selectfont\itshape
% 20080211 CAUSAL PRODUCTIONS
% separate superscript on following line from affiliation using narrow space
%$^{\#}$\,First-Third Department, First-Third University\\ %  Removed for anonymous submission
%Address Including Country Name\\ %  Removed for anonymous submission
\,\\ 
\\
\fontsize{9}{9}\selectfont\ttfamily\upshape
%
% 20080211 CAUSAL PRODUCTIONS
% in the following email addresses, separate the superscript from the email address
% using a narrow space \,
% the reason is that Acrobat Reader has an option to auto-detect urls and email
% addresses, and make them 'hot'.  Without a narrow space, the superscript is included
% in the email address and corrupts it.
% Also, removed ~ from pre-superscript since it does not seem to serve any purpose
%$^{1}$\,first.author@first-third.edu\\ % Removed for anonymous submission
%$^{3}$\,third.author@first-third.edu %  Removed for anonymous submission
\,Anonymous VCIP Submission\\
\,Paper ID:

% add some space between email and affil
\vspace{1.2mm}\\
\fontsize{10}{10}\selectfont\rmfamily\itshape
% 20080211 CAUSAL PRODUCTIONS
% separated superscript on following line from affiliation using narrow space \,
% $^{*}$\,Second Company\\ %  Removed for anonymous submission
% Address Including Country Name\\ %  Removed for anonymous submission
\,\\ 
\\

\fontsize{9}{9}\selectfont\ttfamily\upshape
% 20080211 CAUSAL PRODUCTIONS
% removed ~ from pre-superscript since it does not seem to serve any purpose
%$^{2}$\,second.author@second.com %  Removed for anonymous submission
\,
}
%
\begin{document}
\maketitle

% INCLUDES COPYRIGHT NOTICE: one of three copyright notice should be included. Uncomment the appropriate line below, according to the authors affiliation:
\begin{figure}[b]
\parbox{\hsize}{\em
%information about the event:
%IEEE VCIP'14, Dec. 7 - Dec. 10, 2014, Valletta, Malta.

%copyright notice: one of three copyright notices below should be included. Uncomment the appropriate line, according to the authors affiliation:
%000-0-0000-0000-0/00/\$31.00 \ \copyright 2014 IEEE.
%U.S. Government work not protected by U.S. copyright.
%???-?-????-????-?/10/\$??.?? \copyright 2014 Crown.
}\end{figure}


\begin{abstract}
keep sharp features like edges while keep the evenness of points
\\[1\baselineskip]
\end{abstract}


% NOTE keywords are not used for conference papers so do not populate them
\begin{keywords}
Include at least 5 keywords or phrases
\end{keywords}
%


\section{Introduction}
%
what's simplification and what's the meaning of it

\section{Related Work}

\subsection{Point Cloud Simplification}

what's done

advantages and disadvantages

\begin{itemize}
\item	Top = 19mm (0.75")
\item	Bottom = 43mm (1.69")
\item	Left = Right = 14.32mm (0.56")
\end{itemize}

\subsection{Graph Signal Processing}

what's GSP and how to construct graph

\section{Problem Formulation}

Here, we describe point cloud simplification as a process of 
resampling of the point cloud: given a point cloud $P$ with 
$|P|=N$, find a point cloud $P' \subset P$ with $|P'|=M<N$. 
We define the simplification rate $\alpha=M/N$. 

For convenience, we represent the point cloud with $N$ points
and $K$ attributes as $X \in \mathbb R^{N \times K}$, 
where $i$th row represents the $i$th point, denoted as $x_i^T$. 
Attributes can be coordinates, colors and others, $K \ge 3$.
To represent the simplified point cloud, we consider the diagonal 
matrix $\Psi$, called resampling matrix with $\Psi_{ii}=1$ if 
$x_i$ in the simplified point cloud and $\Psi_{ii}=0$ if not. 
Thus, the simplified point cloud can be represented as $\Psi X$.

Our goal is to find the optimal resampling matrix $\Psi$ to keep 
most geometry features of the point cloud while keep the evenness. 
Inspired by Chen[], we use graph filter to extract features of point 
cloud and select points with higher features. We use the random walk 
Laplacian
\begin{center}
$L_0=D^{-1}L=I-D^{-1}W$
\end{center}
% cancel the indent
\noindent{to extract features, which is a high-pass graph filter 
keeping sharp features. Thus, we can represent features of point 
cloud $X$ as $L_0X$ and the remaining features (of the simplified 
point cloud) as $\Psi L_0X$. Now we define the feature loss of 
simplification as}
\begin{center}
$\mathcal{L}_f=\lVert\Psi L_0X-L_0X\rVert^2_F$.
\end{center}
\noindent{Here, we use the F-norm of matrix and set F as 2.}

However, merely using random walk Laplacian will cause the unevenness 
of point cloud because edges with sharp features will be saved sound 
while the surfaces will be neglected, which will cause extreme 
unevenness. To avoid this extreme unevenness, we define a evenness 
term to control the evenness of the simplified point cloud. 

When constructing the graph, we select points within radius $r$ as 
point's neighbour. If we suppose the point cloud is even, the number 
of neighbours of each point should be approximately equal and thus 
we can use $k$-nearest neighbours -- when the point cloud is even, the 
degree of each node is approximately equal.

We use binary matrix $A$ to represent the adjacency of graph i.e 
$A_{ij}=1$ if and only if $x_j$ is one of the neighbours of $x_i$.
Each line of A represent the relation of the point with its neighbours 
and the sum of each line should be $k$. By means of the definition of 
$\Psi$, the adjacency matrix of the simplified point cloud graph can 
be represented as $A\Psi$. Given the simplification rate $\alpha$, the 
number of neighbours in the simplified point cloud graph should be 
approximately equal to $alpha k$ if simplified evenly. So we define the 
evenness term as
\begin{center}
$\mathcal{L}_e=\lVert A\Psi\boldsymbol{1}-
\alpha k\boldsymbol{1}\rVert^2_F$,
\end{center}
\noindent{where $\boldsymbol{1}$ represent the column vector with every 
element equal to 1.}

Now we can formulate the point cloud simplification problem as an 
optimization problem:
\begin{center}
$\min\limits_{\Psi}\mathcal{L}=\mathcal{L}_f+\lambda\mathcal{L}_e=
\lVert\Psi L_0X-L_0X\rVert^2_F+\lambda\lVert A\Psi\boldsymbol{1}-
\alpha k\boldsymbol{1}\rVert^2_F$,\\
s.t. $\Psi_{ii}\in{0,1}, i=1,2,...,N$; $\Psi_{ij}=0, i\neq j$; 
$tr(\Psi)=\alpha N$,
\end{center}
\noindent{where $\lambda$ is a hyper-parameter to keep balance of 
feature and evenness.}



\section{Formulation Optimization}
%\label{sec:page style}

The optimization problem we put forward before is a combinatorial 
optimization problem, which is NP-hard. To simplify the algorithm, 
we relax the constraints to approximate diagonal matrix $\Psi$ and 
$0 \le \Psi_{ij} \le 1$. Then the optimization problem can be 
represented as
\begin{center}
$\min\limits_{\Psi}\mathcal{L}=\lVert\Psi L_0X-L_0X\rVert^2_F+\lambda
\lVert A\Psi\boldsymbol{1}-\alpha k\boldsymbol{1}\rVert^2_F$,\\
s.t. $tr(\Psi)=\alpha N$, $\lVert\Psi\rVert^2_F=\alpha N$.
\end{center}

Now we can use the method of Lagrange multiplier to solve this 
optimization problem. Suppose the Lagrange multipliers for the two 
constraints are $\beta$ and $\gamma$ respectively, the solution will 
be
\begin{center}
$\Psi=(\gamma I+I+\lambda A^2)^{-1}(L_0XX^TL_0^T+\lambda k\alpha AJ-
1/2 \beta I)(L_0XX^TL_0^T+\gamma I+J)^{-1}$,
\end{center}
\noindent{where $I \in \mathbb{R}^{N \times N}$ represent the 
identify matrix and 
$J=\boldsymbol{1 1}^T \in \mathbb{R}^{N \times N}$ 
represent the matrix with every element equal to 1. And we 
heuristically set $\gamma$ and $\beta$ as $1/\alpha-1-2k\lambda$ and 
$-\alpha\gamma(\gamma+1)$ respectively.}

Each line of relaxed $\Psi$, denoted as $\Psi_i^T$ can be regarded as 
weights of point $x_j$ to be selected relevant to other points. We 
sum each element of $\Psi_i$ and define it as the priority of point $x_j$.
Then we sort the points according to their priority and select the top 
$\alpha$ points as the simplified point cloud.



\section{Experiment Results}

Our algorithm depends on matrix operations, which is time and storage 
expensively. To solve this problem, we divide the point cloud into 
small grids first, and then simplify each grid respectively. For better 
performance, there is small overlapping between adjacent grid.

\subsection{Results compared to previous algorithms}

bunny???
Alice???
dragon???
monster???

time????????
performance: visually(manifold) and quantitatively(error)

\subsection{Results on excessively large point clouds}

landscape???

\section{Conclusion}

advantage:
avoidance of normals
graph filter, local \& global
sharp features(edges) while even

disadvantage: small holes!

Recommended font sizes are shown in Table \ref{tab:font-sizes}.

/

Title must be in 24 pt Regular font.  Author name must be in 11
pt Regular font.  Author affiliation must be in 10 pt Italic.
Email address must be in 9 pt Courier Regular font.

\begin{table}[!h]
\centering

    \caption{Font Sizes for Papers}     % NOTE!  caption goes _before_ the table contents !!
    \label{tab:font-sizes}

    \begin{small}
    \begin{tabular}{|l|l|l|l|}
    \hline
    {\bfseries Font} & \multicolumn{3} {c|} {\bfseries Appearance (in Times New Roman or Times} \\
    \cline{2-4}
    {\bfseries Size} & {\bfseries  Regular}         & {\bfseries Bold}     & {\bfseries Italic}           \\
    \hline
    8         & table caption (in	&		& reference item	 \\
              & Small Caps),		&		& (partial)		\\
              &	figure caption,		&		&			\\
              &	reference item		&		&			\\
    \hline
    9         & author email address	& abstract body & abstract heading	\\
              &	 (in Courier),		&		&    (also in Bold)	 \\
              &	cell in a table		&		&			\\
    \hline
    10        & level-1 heading  (in 	&		& level-2 heading,      \\
              & Small Caps),		&		& level-3 heading,	 \\
              &	paragraph		&		& author affiliation	 \\
    \hline
    11        &	author name		&		&			\\
    \hline
    24        & title			&		&			\\
    \hline
    \end{tabular}
    \end{small}
\end{table}

All title and author details must be in single-column format and
must be centered.

Every word in a title must be capitalized except for short minor
words such as ``a'', ``an'', ``and'', ``as'', ``at'', ``by'', ``for'', ``from'',
``if'', ``in'', ``into'', ``on'', ``or'', ``of'', ``the'', ``to'', ``with''.

Author details must not show any professional title (e.g.
Managing Director), any academic title (e.g. Dr.) or any
membership of any professional organization (e.g. Senior
Member IEEE).

To avoid confusion, the family name must be written as the
last part of each author name (e.g. John A.K. Smith).

Each affiliation must include, at the very least, the name of
the company and the name of the country where the author is
based (e.g. Causal Productions Pty Ltd, Australia).

Email address is compulsory for the corresponding author.


\subsection{Section Headings}

No more than 3 levels of headings should be used.  All headings must
be in 10pt font.  Every word in a heading must be capitalized except
for short minor words as listed in Section \ref{sec:title and author
details}.

\subsubsection{Level-1 Heading}

A level-1 heading must be in Small Caps, centered and numbered using
uppercase Roman numerals.  For example, see heading ``\ref{sec:page
style}. Page Style'' of this document.  The two level-1 headings which
must not be numbered are ``Acknowledgment'' and ``References''.

\subsubsection{Level-2 Heading}

A level-2 heading must be in Italic, left-justified and numbered using
an uppercase alphabetic letter followed by a period.  For example, see
heading ``C. Section Headings'' above.

\subsubsection{Level-3 Heading}

A level-3 heading must be indented, in Italic and numbered with an
Arabic numeral followed by a right parenthesis. The level-3 heading
must end with a colon.  The body of the level-3 section immediately
follows the level-3 heading in the same paragraph.  For example, this
paragraph begins with a level-3 heading.

\subsection{Figures and Tables}

Figures and tables must be centered in the column.  Large figures and
tables may span across both columns.  Any table or figure that takes
up more than 1 column width must be positioned either at the top or at
the bottom of the page.

Graphics may be full color.  All colors will be retained on the CDROM.
Graphics must not use stipple fill patterns because they may not be
reproduced properly.  Please use only SOLID FILL colors which contrast
well both on screen and on a black-and-white hardcopy, as shown in
Fig.  \ref{fig:sample_graph}.

\begin{figure}[h]
	\centerline{\psfig{figure=fig_1.eps,width=68.7mm} }
	\caption{A sample line graph using colors which contrast well both on screen and on a black-and-white hardcopy}
	\label{fig:sample_graph}
\end{figure}

Fig. \ref{fig:lores-photo} shows an example of a low-resolution image
which would not be acceptable, whereas Fig.  \ref{fig:hires-photo}
shows an example of an image with adequate resolution.  Check that the
resolution is adequate to reveal the important detail in the figure.

Please check all figures in your paper both on screen and on a
black-and-white hardcopy.  When you check your paper on a
black-and-white hardcopy, please ensure that:

\begin{itemize}
\item	the colors used in each figure contrast well,
\item	the image used in each figure is clear,
\item	all text labels in each figure are legible.
\end{itemize}

\begin{figure}[h]
	\centerline{\psfig{figure=lores_photo.eps,height=64.54mm} }
	\caption{Example of an unacceptable low-resolution image}
	\label{fig:lores-photo}
\end{figure}

\begin{figure}[h]
	\centerline{\psfig{figure=hires_photo.eps,height=64.54mm} }
	\caption{Example of an image with acceptable resolution}
	\label{fig:hires-photo}
\end{figure}

\subsection{Figure Captions}

Figures must be numbered using Arabic numerals.  Figure captions must
be in 8 pt Regular font.  Captions of a single line (e.g. Fig.
\ref{fig:lores-photo}) must be centered whereas multi-line captions
must be justified (e.g. Fig.  \ref{fig:sample_graph}).  Captions with
figure numbers must be placed after their associated figures, as shown
in Fig. \ref{fig:sample_graph}.

\subsection{Table Captions}

Tables must be numbered using uppercase Roman numerals.  Table
captions must be centred and in 8 pt Regular font with Small Caps.
Every word in a table caption must be capitalized except for short
minor words as listed in Section \ref{sec:title and author details}.
Captions with table numbers must be placed before their associated
tables, as shown in Table \ref{tab:font-sizes}.

\subsection{Page Numbers, Headers and Footers}

Page numbers, headers and footers must not be used.

\subsection{Links and Bookmarks}

All hypertext links and section bookmarks will be removed from
papers during the processing of papers for publication.  If you
need to refer to an Internet email address or URL in your paper,
you must type out the address or URL fully in Regular font.

\subsection{References}

The heading of the References section must not be numbered.
All reference items must be in 8 pt font.  Please
use Regular and Italic styles to distinguish different fields as
shown in the References section. Number the reference items
consecutively in square brackets (e.g. \cite{IEEEexample:book}).

When referring to a reference item, please simply use the
reference number, as in \cite{IEEEexample:bookwithseriesvolume}.
Do not use �Ref. \cite{IEEEexample:article_typical}� or
�Reference \cite{IEEEexample:article_typical}� except at the
beginning of a sentence, e.g.  ``Reference
\cite{IEEEexample:article_typical} shows �''.  Multiple
references are each numbered with separate brackets (e.g.
\cite{IEEEexample:bookwithseriesvolume},
\cite{IEEEexample:article_typical},
\cite{IEEEexample:confwithpaper}--[6]).

Examples of reference items of different categories shown in the
References section include:

\begin{itemize}
\item	example of a book in \cite{IEEEexample:book}
\item	example of a book in a series in \cite{IEEEexample:bookwithseriesvolume}
\item	example of a journal article in \cite{IEEEexample:article_typical}
\item	example of a conference paper in \cite{IEEEexample:confwithpaper}
\item	example of a patent in \cite{IEEEexample:uspat}
\item	example of a website in \cite{IEEEexample:IEEEwebsite}
\item	example of a web page in \cite{IEEEexample:shellCTANpage}
\item	example of a databook as a manual in \cite{IEEEexample:motmanual}
\item	example of a datasheet in \cite{IEEEexample:datasheet}
\item	example of a master's thesis in \cite{IEEEexample:masterstype}
\item	example of a technical report in \cite{IEEEexample:techreptype}
\item	example of a standard in \cite{IEEEexample:standard}
\end{itemize}

% the following command shrinks the final page to force the columns to
% be balanced.  You will need to adjust the value according to the
% appearance of your last page.  Start by setting the value to 0mm
% and slowly increase it until the columns balance.  Alternatively,
% use balance.sty to do the job.
\enlargethispage{-62mm}

\section{Conclusion}

The version of this template is V2.  Most of the formatting
instructions in this document have been compiled by Causal Productions
from the IEEE LaTeX style files.  Causal Productions offers both A4
templates and US Letter templates for LaTeX and Microsoft Word.  The
LaTeX templates depend on the official IEEEtran.cls and IEEEtran.bst
files, whereas the Microsoft Word templates are self-contained.
Causal Productions has used its best efforts to ensure that the
templates have the same appearance.

Causal Productions permits the distribution and revision of these
templates on the condition that Causal Productions is credited in the
revised template as follows: ``original version of this template was
provided by courtesy of Causal Productions
(www.causalproductions.com)''.

\vfill\eject

\section*{Acknowledgment}

The heading of the Acknowledgment section and the References section
must not be numbered.

Causal Productions wishes to acknowledge Michael Shell and other
contributors for developing and maintaining the IEEE LaTeX style files
which have been used in the preparation of this template. 

Are acknowledgements OK? No. Please omit them in the review copy 
and leave them for the final copy.

\bibliographystyle{IEEEtran}

\bibliography{IEEEabrv,IEEEexample}

\end{document}
